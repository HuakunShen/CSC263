\documentclass[10pt]{article}
\usepackage{amsmath}
\usepackage{geometry}
\usepackage{listings}
 \geometry{
 a4paper,
 total={170mm,257mm},
 left=20mm,
 top=20mm,
 }
\begin{document}
\noindent \textbf{CSC263 A2}\\
January 31, 2019\\
Xu Wang, Jiatao Xiang, Huakun Shen
\section*{Question 1}
Written by Huakun Shen, Checked by Jiatao Xiang and Xu Wang
\begin{enumerate}
\item[a.] Increase the key of a given item $x$ in a binomial max heap $H$ to
become $k$\\
\textbf{Increase(H, x, k):}
\begin{enumerate}
\item[1)]Change the key value of $x$ to $k$, assume $k>x$
\item[2)]While the \underline{\textit{$x$ has a parent ($x$ is not a root)}} and \underline{\textit{the key of $x$'s parent is smaller than $k$}}:\\
Do step 3
\item[3)]switch the position of $x$ and its parent
\end{enumerate}
\textbf{Worst Case Runtime:}\\
Assume $H$ has $n$ nodes, $n=<b_t,b_{t-1},...,b_0>_2$,		where $t=\left\lfloor log_2n\right\rfloor$\\
$H=F_n:\text{ }<\text{all trees } B_i \text{ such that bit }b_i = 1>$\\
The largest binomial tree in $H$ is $B_t$, whose number of node is $2^t$, and height is $t$.\\
Then, in the worst case, $x$ needs to be switched at most $t$ times to become the root of its binomial tree.\\
Since $t=\left\lfloor log_2n\right\rfloor$, $x$ would be switched at most $\left\lfloor log_2n\right\rfloor$ times.\\
$RT_{WC}=\mathcal{O}(log_2n)$



\item[b.]Delete a given item $x$ from a binomial max heap $H$.\\
\textbf{Remove(H, x):}
\begin{enumerate}
\item[1.]While $x$ is not the root of its binomial tree: \\do step 2
\item[2.]$Increase(H, x, \text{key of }x's\text{ parent} + 1)$
\item[3.]Locate the maximum node $m$ of $H$, which is one of the roots of the binomial trees in the binomial heap\\
Let's say $B_i$ is the binomial tree that contains $m$, isolate $B_i$ and create a new binomial heap $U$\\
$U=H-B_i$
\item[4.]Delete the root node which was located in step 3, and make $S$ a new binomial tree of the result\\
$S=B_i-m$ \null\hfill ($m$ is the root of $B_i$, the max node in $H$)
\item[5.]$H\leftarrow Union(U,S)$
\end{enumerate}
\textbf{Worst Case Runtime:}
\begin{itemize}
\item
The goal of the step 1, 2 is to move and make $x$ the root of its binomial tree. Every time $x$'s key is increased to ($x$'s parent's key $+1$), $x$ switches with its parent. As explained in (a), a binomial heap with $n$ nodes has a maximum binomial tree of height $\lfloor log_2n\rfloor$, thus it takes at most $\lfloor log_2n\rfloor$ basic operations to make $x$ the root of its binomial tree. $RT_1=\mathcal{O}(log_2n)$
\item
Step 3 searches through the root of every binomial tree in the binomial heap to locate the maximum node in $H$\\
For a binomial heap with $n$ nodes, it has $\lfloor log_2n\rfloor$ binomial trees. Thus it takes at $\lfloor log_2n\rfloor$ steps to locate the binomial tree with the maximum node. $RT_2=\mathcal{O}(log_2n)$
\item
Step 4 deletes the root of a binomial tree, which takes constant time. $RT_3=\mathcal{O}(1)$
\item 
Step 5 makes $H$ the union of the results from step 3 and step 4, which takes $RT_3=\mathcal{O}(log_2n)$ of time
\item
$RT_{WC} = RT_1 + RT_2 + RT_3 + RT_4 = \mathcal{O}(log_2n)$
\end{itemize}
In brief, the algorithm of $remove(H, x)$ is:
\begin{enumerate}
\item $increase(H,x,\infty)$
\item $extract\_max(H)$ \null\hfill (The sum of step 3 - 5)
\end{enumerate}
\end{enumerate}
\newpage
\section*{Question 2}
Written by Jiatao Xiang, Checked by Xu Wang and Huakun Shen
\begin{enumerate}
\item
Our \textbf{\textit{SuperHeap}} is based on \textit{Binomial Max Heap and Min Heap} with a little modification.\\
\textit{Binomial Max Heap} is basically symmetric to \textit{Binomial Min Heap}, and we will build our \textbf{\textit{SuperHeap}} based on them.\\
We also make use of our solution from \textbf{Question 1}, the \textit{Remove($H, x$)} function (which is  built on \textit{Binomial Max Heap}).\\
\textbf{Idea:} We use a max heap and a min heap in our data structure to make super heap, which can trace both min and max values.\\
\textbf{How:} We store extra information (an attribute) called \textbf{twinValue} in each node,``a pointer to the twin node in the other heap which has the same key."\\

\item Implementation of methods\\
Let's call the SuperHeap $SH$, and the max heap $MaxH$, the min heap $MinH$.
\begin{enumerate}
\item \textit{Merge($D, D'$):}\\
We discussed in lecture that the merge function of Binomial Max Heaps is very similar to that of Max Binomial Heaps except we keep max value on the top in Max Binomial Heap. In SH, Merge($D, D'$) is implemented by merging two MaxHs from D and D' and two MinH from D and D'. This takes $\mathcal{O}(log_2n)+\mathcal{O}(log_2n)\rightarrow\mathcal{O}(log_2n)$.
\item \textit{Insert(k):} \\
When we perform insert operation, we have to insert the same node k to both MinH and MaxH, which means we need two node with the same value and they point to each other with their \textbf{twinValue} attribute. The worst case run time takes $\mathcal{O}(log_2n)+\mathcal{O}(log_2n)\rightarrow\mathcal{O}(log_2n)$.

\item \textit{ExtractMax():}\\
It's exactly symmetric to \textit{ExtractMin()} in \textit{Binomial Min Heap} that we discussed during lecture, except we need to remove the max value in both min heap and max heap in our data structure.
\begin{enumerate}
\item Extract the max value in $MaxH$ by comparing the root of every tree in $MaxH$, which costs $\mathcal{O}(log_2n)$. Let's call the node extracted $Node$.
\item Then, find corresponding max node in $MinH$ using the attribute \textbf{twinValue} of $Node$, which costs constant time.
\item Finally, we perform \textit{Remove(MinH, Node.twinValue)} to \textit{MinH} to delete the corresponding node. The remove function comes from part b of Question 1, which has a runtime of $\mathcal{O}(log_2n)$. Note that, in Question 1, the $Remove$ function is for Max Heap, here we use the Min-Heap version of Remove, which is exactly the opposite: decrease the key of a node to $-\infty$ and perform $ExtractMin()$ to $MinH$.
\end{enumerate}

Thus, \textit{ExtractMax()} takes $\mathcal{O}( log_2n)$ of time.

\item \textit{ExtractMin():}
ExtractMin function is almost the same as Extract Max, the only difference is that, this time, we extract the min value from both heaps.
\begin{enumerate}
\item Extract the min value from $MinH$ by comparing the root of every tree in $MinH$, which costs $\mathcal{O}(log_2n)$. Let's call the node extracted $Node$.
\item then, we find the corresponding min node in the $MaxH$ using the twinValue of $Node$, which costs constant time.
\item Finally, perform $Remove(MaxH, Node.twinValue)$ function to $MaxH$ to delete the corresponding node. The Remove function comes from part b of Question 1, which has runtime of $\mathcal{O}(log_2n)$.
\end{enumerate}

Thus, \textit{ExtractMin()} takes $\mathcal{O}( log_2n)$ of time.
\end{enumerate}
\end{enumerate}
\newpage

\section*{Question 3} Written by Xu Wang, Checked by Jiatao Xiang and Huakun Shen
\begin{itemize}
\item[a)]
\begin{lstlisting}[frame=single]
PathLengthFromRoot(root, k){
    if(key(root) == k){
        return 1;    
    }
    if(k > key(root)){
		return 1 + PathLengthFromRoot(rchild(root), k);
    }else{
		return 1 + PathLengthFromRoot(lchild(root), k);
    }

}
\end{lstlisting}
\textbf{Description:} The algorithm recursively tracks down to key k by comparing its value with the current root and count the number of calls taken. The base case is when value of root is equal to value of key.\\
\textbf{Worst-Case Time Complexity:} the height of the BST is h. Each step of the algorithm will increase depth by 1 and loop at most h times which is the height of the BST and thus is $\mathcal{O}(h)$.
\item[b)]
\begin{lstlisting}[frame=single]
FCP(root, k, m){
    if(k <= key(root) <= m || m <= key(root) <= k){
		return root;    
    }else if(k < key(root) && m < key(root)){
		return FCP(lchild(root), k, m);    
    }else{
		return FCP(rchild(root), k, m);    
    }
}
\end{lstlisting}
\textbf{Description:} The algorithm finds the furthest parent of k and m by looking for the first node that $k (m)\leq Key() \leq m (k)$.\\
\textbf{Worst-Case Time Complexity:} Each step of the algorithm will increase depth by 1 and it will loop at most h times which is the height of the BST and thus is $\mathcal{O}(h)$.

\item[c)]
\begin{lstlisting}[frame=single]
IsTAway(root, k, m, t){
    ParentNode = FCP(root, k, m);
    Path1 = PathLengthFromRoot(root, k);
    Path2 = PathLengthFromRoot(root, m);
    return (Path1 + Path2 <= t);
}
\end{lstlisting}
\textbf{Description:} The algorithm finds the furthest node(let's say node n) from the root where the sub-tree rooted at this node contains k and m. Then using the method in part a) to calculate the path to k and m from n. Calculate the sum of two paths which is path length between k and m and compare it with t, then return the result.\\
\textbf{Worst-Case Time Complexity:} the worst-case run time of FCP and PathLengthFromRoot is $\mathcal{O}(h)$ and thus the total runn time of IsTAway will also be $c_1$*h, where $c_1$ is a constant, thus it is $\mathcal{O}(h)$.
\end{itemize}
\end{document}






































