\documentclass[10pt]{article}
\usepackage{amsmath}
\usepackage{geometry}
 \geometry{
 a4paper,
 total={170mm,257mm},
 left=20mm,
 top=20mm,
 }
\begin{document}
\noindent \textbf{CSC263 A2}\\
January 31, 2019\\
Xu Wang, Jiatao Xiang, Huakun Shen
\section*{Question 1}
\begin{enumerate}
\item[a.] Increase the key of a given item $x$ in a binomial max heap $H$ to
become $k$\\
\textbf{Increase(H, x, k):}
\begin{enumerate}
\item[1)]Change the key value of $x$ to $k$, assume $k>x$
\item[2)]While the \underline{\textit{$x$ has a parent ($x$ is not a root)}} and \underline{\textit{the key of $x$'s parent is smaller than $k$}}:\\
Do step 3
\item[3)]switch the position of $x$ and its parent
\end{enumerate}
\textbf{Worst Case Runtime:}\\
Assume $H$ has $n$ nodes, $n=<b_t,b_{t-1},...,b_0>_2$,		where $t=\left\lfloor log_2n\right\rfloor$\\
$H=F_n:\text{ }<\text{all trees } B_i \text{ such that bit }b_i = 1>$\\
The largest binomial tree in $H$ is $B_t$, whose number of node is $2^t$, and height is $t$.\\
Then, in the worst case, $x$ needs to be switched at most $t$ times to become the root of its binomial tree.\\
Since $t=\left\lfloor log_2n\right\rfloor$, $x$ would be switched at most $\left\lfloor log_2n\right\rfloor$ times.\\
$RT_{WC}=\mathcal{O}(log_2n)$



\item[b.]Delete a given item $x$ from a binomial max heap $H$.\\
\textbf{Remove(H, x):}
\begin{enumerate}
\item[1.]While $x$ is not the root of its binomial tree: \\do step 2
\item[2.]$Increase(H, x, \text{key of }x's\text{ parent} + 1)$
\item[3.]Locate the maximum node $m$ of $H$, which is one of the roots of the binomial trees in the binomial heap\\
Let's say $B_i$ is the binomial tree that contains $m$, isolate $B_i$ and create a new binomial heap $U$\\
$U=H-B_i$
\item[4.]Delete the root node which was located in step 3, and make $S$ a new binomial tree of the result\\
$S=B_i-m$ \null\hfill ($m$ is the root of $B_i$, the max node in $H$)
\item[5.]$H\leftarrow Union(U,S)$
\end{enumerate}
\textbf{Worst Case Runtime:}
\begin{itemize}
\item
The goal of the step 1, 2 is to move and make $x$ the root of its binomial tree. Every time $x$'s key is increased to ($x$'s parent's key $+1$), $x$ switches with its parent. As explained in (a), a binomial heap with $n$ nodes has a maximum binomial tree of height $\lfloor log_2n\rfloor$, thus it takes at most $\lfloor log_2n\rfloor$ basic operations to make $x$ the root of its binomial tree. $RT_1=\mathcal{O}(log_2n)$
\item
Step 3 searches through the root of every binomial tree in the binomial heap to locate the maximum node in $H$\\
For a binomial heap with $n$ nodes, it has $\lfloor log_2n\rfloor$ binomial trees. Thus it takes at $\lfloor log_2n\rfloor$ steps to locate the binomial tree with the maximum node. $RT_2=\mathcal{O}(log_2n)$
\item
Step 4 deletes the root of a binomial tree, which takes constant time. $RT_3=\mathcal{O}(1)$
\item 
Step 5 makes $H$ the union of the results from step 3 and step 4, which takes $RT_3=\mathcal{O}(log_2n)$ of time
\item
$RT_{WC} = RT_1 + RT_2 + RT_3 + RT_4 = \mathcal{O}(log_2n)$
\end{itemize}
In brief, the algorithm of $remove(H, x)$ is:
\begin{enumerate}
\item $increase(H,x,\infty)$
\item $extract\_max(H)$ \null\hfill (The sum of step 3 - 5)
\end{enumerate}
\end{enumerate}
\newpage
\section*{Question 2}
\begin{enumerate}
\item
Our \textbf{\textit{SuperHeap}} is based on \textit{Binomial Max Heap}, with a little modification.\\
\textit{Binomial Max Heap} is basically symmetric to \textit{Binomial Min Heap}, they have the same but inverse implementation, and we will build our \textbf{\textit{SuperHeap}} based on it.\\
We will also make use of our solution from \textbf{Question 1}, the \textit{Remove($H, x$)} function (which is also built on \textit{Binomial Max Heap}.\\
\textbf{Idea:} Always keep the minimum node of each binomial tree in the bottom, leftmost position, and make every node in the tree keep track of it. Then it takes less time to find the min node.\\
\textbf{How:} We store an extra information (attribute) called \textbf{min} in each node,``a pointer to the leftmost(bottom) node in the binomial tree".\\
Let's say we perform all operations on a \textit{Binomial Min Heap} $T$, whose number of nodes$=n$.


\item Implementation of methods
\begin{enumerate}
\item \textit{Merge($D, D'$):}
Similar to the algorithm of \textit{Union} we talked about in class, which takes $\mathcal{O}(log_2n)$ of time, except we perform one extra step, comparing and switching the leftmost node when merging two trees.\\
Merging two heaps is in fact performing ``merging two trees'' a bunch of times. \\
We first describe how to merge two trees. Call the two trees $t1$ and $t2$.
\begin{enumerate}
\item Compare the root of $t_1$ and $t_2$, the greater root becomes the root of the new tree, and let's suppose $t_1$ has the greater root.
\item Then compare the leftmost node of both trees. If the leftmost node of the $t_1$ is smaller, then switch the node with the leftmost node of the $t_2$, to make sure the new tree's min node is in the leftmost position.
\item Note: When switching the two leftmost node, only switch the value, then we don't have to modify other node's \textbf{min} attriubute to point to the leftmost node.
\end{enumerate}
The rest is the same as binary addtion. Comparing and switching the value of two nodes takes constant time. \\
Overall, \textit{Merge($D, D'$)} still takes $\mathcal{O}(log_2n)$ of time.
\item \textit{Insert(k):} 
\begin{enumerate}
\item Make a new binomial tree $T_2$ with a single node with key $k$.
\item \textit{Merge($T,T_2$)} \null\hfill (takes $\mathcal{O}(log_2n)$)
\end{enumerate}
Making a single node takes constant time, and \textit{Merge($T,T_2$)} takes $\mathcal{O}(log_2n)$ of time.
\item \textit{ExtractMax():}\\
It's exactly symmetric to \textit{ExtractMin()} on \textit{Binomial Min Heap} 
\begin{enumerate}
\item Search through the roots of every binomial tree in the binomial heap, call the node ``$max$'' and the tree it belongs to ``$t$''. There are at most $\lfloor log_2n \rfloor$ trees in the heap, so it takes $\mathcal{O}(log_2n)$ of time to find $max$.
\item Let $U=T-t$ \null\hfill ($U$ is the rest of the trees)
\item Let $S=t-max$ \null\hfill ($S$ is the remainders of $t$ after $max$ is removed from it)
\item Let $T=U \cup S$ \null\hfill (This takes $\mathcal{O}(\lfloor log_2n\rfloor)$ of time)
\end{enumerate}
\textit{ExtractMax():} takes $\mathcal{O}(\lfloor log_2n\rfloor)$ of time.

\item \textit{ExtractMin():}
\begin{enumerate}
\item Loop through every tree of $T$, and Compare the minimum(using the \textbf{min} attribute of the root of each tree) node of every tree, and find the $min$ of the heap. \\Let's call the minimum node $x$. \null\hfill (there are at most $\lfloor log_2n \rfloor$ trees, thus takes $\lfloor log_2n \rfloor$ of time)
\item Call \textit{Remove($T, x$)} from \textbf{Question 1} \null\hfill (takes $\lfloor log_2n \rfloor$ of time)
\end{enumerate}
Overall, it takes $\lfloor log_2n \rfloor$ of time to extract the minimum node from a our \textbf{SuperHeap}.
\end{enumerate}
\end{enumerate}
\end{document}






































