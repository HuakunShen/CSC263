\documentclass[10pt]{article}
\usepackage{amsmath}
\usepackage{geometry}
\usepackage{fancybox}
\usepackage{amsfonts}
\usepackage{tikz}
\usepackage{listings}
 \geometry{
 a4paper,
 total={170mm,257mm},
 left=20mm,
 top=10mm,
 bottom=15mm
 }
 
\usepackage{color}
 
\definecolor{codegreen}{rgb}{0,0.6,0}
\definecolor{codegray}{rgb}{0.5,0.5,0.5}
\definecolor{codepurple}{rgb}{0.58,0,0.82}
\definecolor{backcolour}{rgb}{0.95,0.95,0.92}

\linespread{1.3}

\title{CSC263H1 Assignment 4}
\author{Jiatao Xiang, Xu Wang, Huakun Shen}
\date{February 28th, 2019}

\begin{document}

\lstdefinestyle{mystyle}{
    backgroundcolor=\color{backcolour},   
    commentstyle=\color{codegreen},
    keywordstyle=\color{magenta},
    numberstyle=\tiny\color{codegray},
    stringstyle=\color{codepurple},
    basicstyle=\footnotesize,
    breakatwhitespace=false,         
    breaklines=true,                 
    captionpos=b,                    
    keepspaces=true,                 
    numbers=left,                    
    numbersep=5pt,                  
    showspaces=false,                
    showstringspaces=false,
    showtabs=false,                  
    tabsize=4
}
\lstset{style=mystyle}
\maketitle
\section*{Question 1}
\begin{enumerate}
\item[a.] There is a probability of $\frac{k}{n}$ for the algorithm to return \textbf{TRUE} in the first iteration.\\
There are $n$ integers in total, where $k$ of them are $x$. Picking one from them yields $\frac{k}{n}$ of probability to get an $x$.

\item[b.] Geometric Distribution. The probability for the algorithm to return \textbf{TRUE} in $r$ iterations is,
\begin{align*}
P(x\leq r)&=(1-\frac{k}{n})^0\cdot \frac{k}{n}+(1-\frac{k}{n})^1\cdot \frac{k}{n}+...+(1-\frac{k}{n})^{r-1}\cdot \frac{k}{n}\\
&=\Sigma^{r-1}_{i=0}((1-\frac{k}{n})^i\cdot \frac{k}{n})
\end{align*}
Where $x$ is the number iterations where the first success occurs.
\item[c.] When the algorithm is modified with an infinite loop, the expected number of loop iterations is $\frac{n}{k}-1$
\begin{align*}
E(x)&=\frac{1-\frac{k}{n}}{\frac{k}{n}}\\
&=(1-\frac{k}{n})\cdot\frac{n}{k}\\
&=\frac{n}{k}-1\\
E(x)&=\Sigma^\infty_{i=0}\Big(1-\frac{k}{n}\Big)^i\cdot\frac{k}{n}\cdot(i+1)\\
&=\Sigma^\infty_{i=0}\Big(\frac{n-k}{n}\Big)^i\cdot\frac{k}{n}\cdot(i+1)
\end{align*}
\end{enumerate}

\section*{Question 2}
For the sake of writing better-to-understand algorithm, we will make some definitions and assumptions.\\
We use disjoint set (Forest Structure) with weighted union by size and path compression.\\
Assume $constraints$ and $variables$ are 2 lists, where $|constraints|=m, |variables|=n$.\\
Assume each constaint contains 3 attributes: 
\begin{itemize}
\item first set (denoted by $constraint[0]$)
\item second set (denoted by $constraint[1]$)
\item type of constaint (denoted by $constraint[2]$, value is whether $equality$ or $inequality$)
\end{itemize}
Index of these list starts at 1.
\begin{lstlisting}[language=Python]
def Algorithm(constraints[], variables[]):
	for i = 1...n:
		Si <- variables[i]	# Make n disjoint sets for each element of variables named S1...Sn
	for constraint in constraints:
		if constaint[2] is equality:
			Union(constraint[0], constraint[1])
	for constraint in constraints:
		if constaint[2] is inequality:
			if Find(constraint[0]) == Find(constraint[1]):
				return Nil
	int i = 0
	for s in sets that are not empty:
		for element in s:
			element = i		
		i++
	return variables
\end{lstlisting}
We use disjoint set for this algorithm.\\
\textbf{Runtime Analysis:}
\begin{enumerate}
\item Line 2-3 initializes n disjoint sets, each represents a variable.\\
It takes $\mathcal{O}(n)$ in total.
\item The loop on line 4 runs $m$ iterations, but since there are $n$ elements, there are at most $(n-1)$ unions.
\item The loop on line 7 runs $m$ iterations, since there are $m$ constraints, there are at most $2m$ $Find$'s (line 9 calls $Find$ twice).
\item For the loop on line 12, although it's a nested loop, since there are $n$ elements, and the nested loop only traverse through every one of them, this part takes $\mathcal{O}(n)$ of time.
\end{enumerate}
From lecture, we know that for $\sigma:$ Sequence of $(n-1)$ $Unions$ mixed with $(m\geq n)$ $Finds$, $\sigma$ takes $\mathcal{O}(m\cdot log^*n)$.\\
Part 2 and 3 altoghter has at most $(n-1)$ $Unions$ and $2m$ $Finds$, which takes $\mathcal{O}(2m\cdot log^*n)=\mathcal{O}(2m\cdot log^*n)$,\\
Altogether, the algorithm takes $\mathcal{O}(n+m\cdot log^*n + n)=\mathcal{O}(n+m\cdot log^*n)<\mathcal{O}(mn)$.


\end{document}